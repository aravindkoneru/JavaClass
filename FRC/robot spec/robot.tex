\documentclass[11pt,fleqn]{article}


\usepackage[latin1]{inputenc}
\usepackage{enumerate}
\usepackage[hang,flushmargin]{footmisc}

\setlength{\oddsidemargin}{0px}
\setlength{\textwidth}{460px}
\setlength{\voffset}{-1.5cm}
\setlength{\textheight}{20cm}
\setlength{\parindent}{0px}
\setlength{\parskip}{10pt}

\begin{document}
\begin{center}
{\Huge
\textbf{Robot Spec}
}\\
\end{center}

\section*{Description}
In order to acquaint you with the wpilib and writing robot code in java, we have made a ``fake'' robot
for you to work with. All of the parts and descriptions below are representative of the different parts
that could be on the robot as well as the configuration they could be in. By writing the code required
for this fake robot to work, you should gain all the skills necessary to make meaningful contributions
to the final robot code. 

\section*{Specifications}
\begin{itemize}
	\item 
	Talons in CAN configuration in positions from 0-8
	\begin{itemize}
		\item 
		0-5: Drive

		\item 
		6-7: Elevator

		\item 
		8: Intake
	\end{itemize}

	\item 
	Encoders in ports 0-7
	\begin{itemize}
		\item 
		0-5: Drive

		\item 
		6-7: Elevator
	\end{itemize}

	\item 
	Solenoid in port 8 for intake
\end{itemize}

\section*{Expectations}
\textbf{What you put in is what you'll get out}. We will go over some information every lesson and
we will assign homework. Homework will be checked for accuracy. Additionally, we have the expectation
that everyone will be pushing their code frequently to their personal GitHub accounts. If you have questions
at any point during the lesson, please ask and get clarification; we expect people to have questions and
by not asking, you're just hurting yourself. 

\begin{center}
{\huge
\textbf{Questions?}
}
\end{center}

\end{document}

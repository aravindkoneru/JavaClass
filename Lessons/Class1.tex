\documentclass[11pt,fleqn]{article}

\usepackage[cache=false]{minted}
\usepackage[latin1]{inputenc}
\usepackage{helvet}
\usepackage{enumerate}
\usepackage[hang,flushmargin]{footmisc}
\usepackage{amsmath}
\usepackage{amsfonts}
\usepackage{amssymb}
\usepackage{amsthm}

\setlength{\oddsidemargin}{0px}
\setlength{\textwidth}{460px}
\setlength{\voffset}{-1.5cm}
\setlength{\textheight}{20cm}
\setlength{\parindent}{0px}
\setlength{\parskip}{10pt}

\begin{document}
\begin{center}
{\Huge
Class 1
}\\
\end{center}

\begin{center}
\section*{Goals}
\end{center}

\begin{enumerate}[1.]
\item
Variables

\item
\texttt{if} statements

\item 
Methods

\item
Testing our code

\item
Commenting/Good practices
\end{enumerate}

\begin{center}
\section*{Variables}
\end{center}

What are variables? If you know some Algebra or Pre-Algebra, you know that
we use variables in math to represent numbers in our equations. Essentially, a
variable is an unknown value that we are trying to solve for. In programming,
variables are also used to represent values. 

In programming, we use variables all the time to store information that we would
want to access later. There are various forms of information we could store. For
example, we may want to store numbers, words, or even a variety of these things.
The type of information we want to store is usually called the
\textbf{\texttt{type}} of the variable. For example, the variable storing the
phrase ``sit vis vobiscum", would be of \texttt{type} string (in java, words are
referenced as strings). 

Some key words that you need to be aware of when discussing variables:
\begin{enumerate}
\item
\textbf{Declaration}

\item
\textbf{Initalization}

\item
\textbf{type}
\end{enumerate}

We have previous defined type, but we still need to define declaration and
initialization. 

A \textbf{declaration} is defined as the act of creating a new variable, but not
assigning it a value. In Java, we declare variables as follows:

\textit{type} name

Say that we wanted to make a variable to store an integer. We would do that by
doing:

\begin{minted}{java}
int anIntegerVariable;
\end{minted}

In this case, we have \textbf{declared} a new variable named
\texttt{anIntegerVariable} of \textbf{type} \texttt{int}. 

\textbf{Exercise}: Look at the following variables and then break down them down
like we did for \texttt{anIntegerVariable}. 

\begin{minted}{java}
    String myName;
    double pi;
    int aName;
    float someNum;
\end{minted}

So far, we have only covered how to declare a variable. This means that all of
our variables are set to \texttt{null} and we can't use them in our code. In
order to assign values to our variables we must \textbf{intialize} them. We can
initialize variables by setting them equal to some value. For example:

\begin{minted}{java}
    String myName;//declaring the variable
    myName = "Aravind";//initializing the variable
\end{minted}

We first declared the variable (\texttt{myName}) and then we assigned it the
value ``Aravind". This is an example of initalizing a variable. It is important
to note that in java, variables accept their assignments from the right. This
means that 

\begin{minted}{java}
    String myName;
    "Aravind" = myName;
\end{minted}

will throw an error. This is the case for most programming languages you will
encounter.

\textbf{Exercise}: Declare and initialize a variable for each of the following
types: 

\begin{minted}{java}
int
float
double
String
\end{minted}

It gets annoying to declare and initialize a variable on different lines. It's
much easier to do both at the same time. We can easily do this by:

\begin{minted}{java}
    int myAge = 17 //declaring and initializing a variable at the same time
\end{minted}

Note that it is common practice to declare and initialize variables on the same
line. There are situations where it may not be possible to do both, but when you
can, you should.

Now that we know about variables and how to use them, we can start doing some
basic tasks in java. I'm a strong believer in application, so many of my example
will come in the form of code. Here is the first example:

\begin{minted}{java}
int num1 = 10;
int num2 = 20;
int num3 = num1 * num2;//What are we doing here?
int num4 = num3 / num1;//Is num4 always equal to num2?
\end{minted}

In this example, we are introduced to something we haven't seen before: defining
a variable in terms of other variables. In our example, \texttt{num3} is defined
to be the product of \texttt{num1} and \texttt{num2}.Although we will always
know the numerical values of \texttt{num1} and \texttt{num2}, we won't always
know the numerical value of \texttt{num3}. Sure when \texttt{num1} and
\texttt{num2} are small, we can calculate the value of \texttt{num3}, we won't
be able to do so when \texttt{num1} and \texttt{num2} are large. The beauty of
this situation is that we always know what the value of \texttt{num3} is in
relation to that of \texttt{num1} and \texttt{num2}, but we may not always know
the numerical value. 

\textbf{Exercise}: Initialize two \texttt{String} variables with your first and
last name. Make a third variable that will store your entire name, but you must
define the third variable in terms of the first two. 

\begin{center}
\section*{\texttt{if} Statements}
\end{center}

Just like in real life, we need to be able to make decisions in our code. What
if we only want the robot to turn when a button is pushed or if a joystick is
being used? In situations like this, we can use \texttt{if} statements to
control the flow of our code. The way that an \texttt{if} statement works is
pretty simple: we put a condition in the code, and if that condition is met,
then we run a different set of code. Here is how we would construct an
\texttt{if} statement:

\begin{minted}{java}
    if(/*condition goes here*/){
        //code goes here
    }
\end{minted}

A condition can be any logical operation that yields a \textbf{boolean}. A
\textbf{boolean} can only either be true or false. Some examples of boolean
operators in java are $>= (\ge), <= (\le), \text{ and } ==$. The code within an
\texttt{if} statement will execute if and only if the condition is met (true).  
Say for example we have the following code:

\begin{minted}{java}
int a = 10;
int b = 12;
int c = a + b;

if(c > 20){
    System.out.println("Greater than 20!");
}
\end{minted}

Let's take a close look at what the code is doing in this situation. We declared
three \texttt{int} variables: a,b, and c. Next, we used an \texttt{if} statement
to check if c was greater than 20. Since \texttt{c = 12 + 10}, we know that
\texttt{c} = 22. This means that \texttt{c > 20} and that the condition is
satisfied. This means that the code would print ``Greater than 20!". 

\textbf{Exercise}: What would happen if a = 5 and b = 10? Would the code print
``Greater than 20!"?

So know how to write simple \texttt{if} statements now, but what if we want the
code to do something else if the condition is not met. Say for example, that we
want the code to tell us if the value of \texttt{c} is less than 20? We can use
the \texttt{else} statement for these situations. An \texttt{else}
statement is used after an \texttt{if} statement like so:

\begin{minted}{java}
    if(/*condition here*/){
        //run this code if the condition was met
    } else{
        //run this code if the condition was not met
    }
\end{minted}

So if the condition in the \texttt{if} statement is not satisfied, then the code
in the else statement will run. Think of it as an ultimatum, if something does
not happen, then this will happen. If you use an \texttt{if-else} block of code,
then you can always be sure that either the \texttt{if} statement code will run
or the \texttt{else} statement code will run. Let's expand our previous example
to print out if \texttt{c} is less than 20:

\begin{minted}{java}
    int a = 10;
    int b = 12;
    int c = a + b;

    if(c > 20){
        System.out.println("Greater than 20!");
    } else{
        System.out.println("Less than 20")
    }
\end{minted}

Now, if the value of \texttt{c} is less than 20, the code will print ``Less than
20". 

\textbf{Exercise}: What happens if \texttt{c} = 20? Does the code do anything at
all?

Thinking about the last exercise probably gave you a headache since it shows
that we have a fault in our logic. If c = 20, then the code will print out
``Less than 20" because that is how an \texttt{if-else} code block works. Since
20 is not less than 20, we go to the else and print out a false statement. This
is a good example of ``dumb programming". Computers are dumb by design and they
can only do what you tell them to. It is up to you to determine edge cases like
this in you code and to accommodate for them. Situations like this arise very
often and they can cause your code to not work like it should. 

Luckily, we have a solution to this problem. We can use the \texttt{else if}
command to check to see if c = 20. The \texttt{else if} comes right after an
\texttt{if} statement and before an \texttt{else} statement. Basically an
\texttt{else if} says that if the condition in the \texttt{if} is not met, then
check to see if the condition in the \texttt{else if} is met. If that condition
is also not satisfied, then continue on to the \texttt{else} statement. Here is
an example of how an \texttt{else if} would work:

\begin{minted}{java}
    if(/*condition a*/){
        //code to run if condition a is met
    } else if{/* condition b*/}{
        //code to run if condition a not met, but condition b is met
    } else{
        //code to run neither conditions are met
    }
\end{minted}

Now that we know how an \texttt{else if} statement works, let's modify our code
from before to fix the bug that we encountered:

\begin{minted}{java}
    int a = 10;
    int b = 12;
    int c = a + b;

    if(c > 20){
        System.out.println("Greater than 20!");
    } else if{c = 20}{
        System.out.println("Equal to 20");
    } else{
        System.out.println("Less than 20");
    }
\end{minted}

Using this final version, we have fixed all of our bugs.

\textbf{Exercise}: Make a program that will compare two numbers to each other
and print out which one is greater. If they are equal, print that the two
numbers are equal. Make sure to use variables. 

\begin{center}
\section*{Methods}
\end{center}

Now that we have a solid understand of variables and \texttt{if} statements we can
begin putting them together in larger segments of code. Since Java is an
object-oriented programming language, we can create blocks of code called
methods. Basically, a method accepts some input or a list of \textbf{parameters}
and then \textbf{returns} a single piece of information. In java the type of the
information that the method will return is pre-defined, but there are other
languages where this is not the case. It is also important to note that a method
can only return information one time and can only return one piece of
information. Let's look at how we would declare a method and break it down into
simple terms:

\begin{minted}{java}
public int mysteryMethod(int a, String b){
    //code goes here
}
\end{minted}

\begin{itemize}
    \item
        \textbf{Modifier}: public

    \item
        \textbf{return type}: int

    \item
        \textbf{Parameters}: (int a, String b)
\end{itemize}

A \textbf{modifier} describes the different forms that a method can take. Making a method
public means that it can be accessed from anywhere. You can call it from another
class and it can always be accessed. We will mostly be working with methods that
are public, so it is important that you understand how to create public methods.

Previously I said that Java requires you return a pre-defined type. The
\textbf{return type} is where you define what type you will be returning. Say
that you want your function to return an \texttt{int} value, then you would make
the \textbf{return type} int. In this situation, our \texttt{mysteryMethod} will
return an \texttt{int}. 

Methods are powerful since they are a set of instructions that we can apply to
any code. \textbf{Parameters} are considered to be ``input" and we need them to
be defined each time the method is called. In this situation, whenever
\texttt{mysteryMethod} is called, it need to be given an \texttt{int} variable
and \texttt{String} variable. If we define a method to require
\textbf{parameters}, it is mandatory that you provide those parameters when you
call the method, otherwise, you will get an error. 

Let's write a method that compares two numbers:

\begin{minted}{java}
    public int compareTo(int a, int b){
        if(a == b){
            return 0;
        } else if{a > b}{
            return 1;
        } else{
            return -1;
        }
    }
\end{minted}

Let's break this method down and look at what it is doing. There are a few
essential questions you need to ask yourself when reading a method to determine
its function:

\begin{enumerate}[1.]
\item
What is the name of the method?

\item
What is the return type of the method?

\item
What is the modifier of the method?

\item
What are the parameters and how are they used?

\item
What is the method trying to accomplish (read the code within the method)?
\end{enumerate} 

The first four questions have to do with how we declare the method.

\textbf{Exercise}: Answer the first four questions about the sample method.

Alright now we are faced with the question of what  this method doing? If
we look closely at the code we are essentially just comparing two numbers and
returning either 1, 0, or -1 depending on the relationship between a and b.

When looking at this method, we are faced with a keyword that we have not seen
before: \texttt{return}. Previously, I said that every method can only return a
single piece of information. The \texttt{return} keyword ensures that you are
giving a value back when the method is called. I understand this is a little
abstract, but try to follow me through this example. Say that we have the
following method:

\begin{minted}{java}
public int addThis(int a, int b){
    return a+b;
}
\end{minted}  

Suppose that we call this method in the following manner:

\begin{minted}{java}
int sum = addThis(7,9);
\end{minted} 
 
This expression is the equivalent of saying:

\begin{minted}{java}
int sum = 16;
\end{minted}

Hopefully you understand what the \texttt{return} keyword does. If you don't
feel free to ask me a question the next time you see me. 

\textbf{Exercise}: Make a method that accepts two integers. If the sum of the
integers are greater than 50, return the sum of the integers. If the product of
the two numbers is greater than 20 return the product of the two integers. If
neither of these conditions is met, return 0.

\begin{center}
\section*{Testing our Code}
\end{center}

We currently have all the tools we need to be to able to write some simple code.
Before we can begin writing larger methods and making cool code, we need to
learn a little bit about how classes are organized. 

Say that we make a new \textbf{class} in Android Studio named \texttt{test}. We
can expect to see the following in the class:

\begin{minted}{java}
public class test{

}
\end{minted}

We really can't run any code with just this template. We need to add a special
method in the class in order to run code. It should look like:

\begin{minted}{java}
public class test{
    
    //methods and stuff go here    
    
    public static void main(String[] args){
        
        //tester code goes here    

    }    

}
\end{minted}



\end{document}

\documentclass[11pt,fleqn]{article}

\usepackage[cache=false]{minted}
\usepackage[latin1]{inputenc}
\usepackage{enumerate}
\usepackage[hang,flushmargin]{footmisc}
\usepackage{amsmath}
\usepackage{amsfonts}
\usepackage{amssymb}
\usepackage{amsthm}

\setlength{\oddsidemargin}{0px}
\setlength{\textwidth}{460px}
\setlength{\voffset}{-1.5cm}
\setlength{\textheight}{20cm}
\setlength{\parindent}{0px}
\setlength{\parskip}{10pt}

\begin{document}
\begin{center}
{\Huge
Class 1
}\\
\end{center}

\section*{Goals}
\begin{enumerate}[1.]
\item
Variables

\item
Methods

\item 
\texttt{if} statements

\item
Testing our code

\item
Commenting/Good practices
\end{enumerate}

\section*{Variables}
What are variables? If you know some Algebra or Pre-Algebra, you know that
we use variables in math to represent numbers in our equations. Essentially, a
variable is an unknown value that we are trying to solve for. In programming,
variables are also used to represent values. 

In programming, we use variables all the time to store information that we would
want to access later. There are various forms of information we could store. For
example, we may want to store numbers, words, or even a variety of these things.
The type of information we want to store is usually called the
\textbf{\texttt{type}} of the variable. For example, the variable storing the
phrase ``sit vis vobiscum", would be of \texttt{type} string (in java, words are
referenced as strings). 

Some key words that you need to be aware of when discussing variables:
\begin{enumerate}
\item
\textbf{Declaration}

\item
\textbf{Initalization}

\item
\textbf{type}
\end{enumerate}

We have previous defined type, but we still need to define declaration and
initialization. 

A \textbf{declaration} is defined as the act of creating a new variable, but not
assigning it a value. In Java, we declare variables as follows:

\textit{type} name

Say that we wanted to make a variable to store an integer. We would do that by
doing:

\begin{minted}{java}
int anIntegerVariable;
\end{minted}

In this case, we have \textbf{declared} a new variable named
\texttt{anIntegerVariable} of \textbf{type} \texttt{int}. 

\textbf{Exercise}: Look at the following variables and then break down them down
like we did for \texttt{anIntegerVariable}. 

\begin{minted}{java}
    String myName;
    double pi;
    int aName;
    float someNum;
\end{minted}

So far, we have only covered how to declare a variable. This means that all of
our variables are set to \texttt{null} and we can't use them in our code. In
order to assign values to our variables we must \textbf{intialize} them. We can
initialize variables by setting them equal to some value. For example:

\begin{minted}{java}
    String myName;//declaring the variable
    myName = "Aravind";//initializing the variable
\end{minted}

We first declared the variable (\texttt{myName}) and then we assigned it the
value ``Aravind". This is an example of initalizing a variable. It is important
to note that in java, variables accept their assignments from the right. This
means that 

\begin{minted}{java}
    String myName;
    "Aravind" = myName;
\end{minted}

will throw an error. This is the case for most programming languages you will
encounter.

\textbf{Exercise}: Declare and initialize a variable for each of the following
types: 

\begin{minted}{java}
int
float
double
String
\end{minted}

It gets annoying to declare and initialize a variable on different lines. It's
much easier to do both at the same time. We can easily do this by:

\begin{minted}{java}
    int myAge = 17 //declaring and initializing a variable at the same time
\end{minted}

Note that it is common practice to declare and initialize variables on the same
line. There are situations where it may not be possible to do both, but when you
can, you should.

Now that we know about variables and how to use them, we can start doing some
basic tasks in java. I'm a strong believer in application, so many of my example
will come in the form of code. Here is the first example:

\begin{minted}{java}
int num1 = 10;
int num2 = 20;
int num3 = num1 * num2;//What are we doing here?
int num4 = num3 / num1;//Is num4 always equal to num2?
\end{minted}

In this example, we are introduced to something we haven't seen before: defining
a variable in terms of other variables. In our example, \texttt{num3} is defined
to be the product of \texttt{num1} and \texttt{num2}.Although we will always
know the numerical values of \texttt{num1} and \texttt{num2}, we won't always
know the numerical value of \texttt{num3}. Sure when \texttt{num1} and
\texttt{num2} are small, we can calculate the value of \texttt{num3}, we won't
be able to do so when \texttt{num1} and \texttt{num2} are large. The beauty of
this situation is that we always know what the value of \texttt{num3} is in
relation to that of \texttt{num1} and \texttt{num2}, but we may not always know
the numerical value. 

\end{document}

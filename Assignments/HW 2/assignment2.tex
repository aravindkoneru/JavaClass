\documentclass[11pt,fleqn]{article}

\usepackage[latin1]{inputenc}
\usepackage{enumerate}
\usepackage[hang,flushmargin]{footmisc}
\usepackage{amsmath}
\usepackage{amsfonts}
\usepackage{amssymb}
\usepackage{amsthm}

\theoremstyle{definition}
\newtheorem{theorem}{Theorem}
\newtheorem{lemma}[theorem]{Lemma}
\newtheorem{corollary}[theorem]{Corollary}
\newtheorem{proposition}[theorem]{Proposition}
\newtheorem{definition}[theorem]{Definition}
\newtheorem{example}[theorem]{Example}

\setlength{\oddsidemargin}{0px}
\setlength{\textwidth}{460px}
\setlength{\voffset}{-1.5cm}
\setlength{\textheight}{20cm}
\setlength{\parindent}{0px}
\setlength{\parskip}{10pt}

\begin{document}
\begin{center}
{\Huge
Homework 2
}\\
\end{center}

\textbf{Overview}: Last class, we worked with the new android system in order to make our robot
move. This is a good start, but all we really did was modify the code that
already existed, we didn't really make any code from scratch. For this
assignment, you should write all the code from scratch. \underline{We will be
deploying your code next class} so be sure to do your best! Try to at least
complete the first 2 problems.

\section*{Assignment Brief}

In the last class, we made the robot move in such a way that one joystick
controlled the throttle  and the other joystick controlled
the direction. For your homework you need to make each side of the robot
dependent on the joystick. For example, if I push the right joystick forward,
then the right drive side should move forward. If I push the left joystick
forward, then the left drive side should move forward. 

Today the turning was controlled by the joystick. For homework you need to make
the turn based on a button. If I push the ``X" button, then the robot should
turn in place in the left direction. You can accomplish this type of turn by
having the left motors moving backward and having the right motors move forward. 

Another turn you will need to implement is an arc-turn. If I
push the ``Y" button, then the robot will arc-turn to the right. This means that
the right motors will not move at all and the left motors will move forward. 

Finally, if I press ``A" at any point in time, everything should stop. 

So in short:

\begin{enumerate}
\item
Left joystick controls left motors and right joystick control right motor

\item
Pressing ``X" makes a pivot turn to the left. 

\item
Pressing ``Y" makes an arc-turn to the right.

\item
Pressing ``A" causes the robot to stop. 
\end{enumerate}

\textbf{Note}: Please make sure that your motor power is not at 1. When turning,
make sure that your motor values are no greater than .5 and no less than -.5.

\begin{center}
Good luck and have fun!
\end{center} 
\end{document}

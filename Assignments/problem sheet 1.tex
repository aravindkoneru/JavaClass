\documentclass[11pt,fleqn]{article}

\usepackage[latin1]{inputenc}
\usepackage{enumerate}
\usepackage[hang,flushmargin]{footmisc}
\usepackage{amsmath}
\usepackage{amsfonts}
\usepackage{amssymb}
\usepackage{amsthm}
\usepackage{listings}

\theoremstyle{definition}
\newtheorem{theorem}{Theorem}
\newtheorem{lemma}[theorem]{Lemma}
\newtheorem{corollary}[theorem]{Corollary}
\newtheorem{proposition}[theorem]{Proposition}
\newtheorem{definition}[theorem]{Definition}
\newtheorem{example}[theorem]{Example}

\setlength{\oddsidemargin}{0px}
\setlength{\textwidth}{460px}
\setlength{\voffset}{-1.5cm}
\setlength{\textheight}{20cm}
\setlength{\parindent}{0px}
\setlength{\parskip}{10pt}

\begin{document}
\begin{center}
{\Huge
Homework Assignment 1
}\\
Basic Java and \texttt{if} statements
\end{center}

\textit{Directions}: Please complete as many of the following as possible. If
you can't complete all of them, it's alright. These questions are meant to
be difficult. At the minimum, complete the first 3 questions. Feel free to
collaborate with your peers on the problems, but please note that \underline{all
submitted work must be your own work}. Also if you don't know a certain method,
(for example, you want to know what \texttt{Math.pow(a,b)} is) feel free to look it
up online. That being said, plagiarism will not tolerated.  

\begin{enumerate}[Q1.]

\item
\texttt{makePerfectSquare(int x)}: A perfect square is a number that has a rational square root. $4, 9, 16, \text{ and } 25$ are examples of perfect squares. Create a method that accepts an integer and then returns the square of that number.

\item
\texttt{isFactor(int a, int b)}: Given two integers \texttt{a} and \texttt{b}, return \texttt{true} if \texttt{b} is a factor of \texttt{a} (the remainder when \texttt{b} divides \texttt{a} is 0). Return \texttt{false} otherwise.

\item
\texttt{isSpecial(double a)}: A number is special if $\frac{a^2}{3}$ is even or if
$a + \frac{a}{2} * 3$ is even. You may assume that $a<1000$. Test cases will not
be provided for this method and you will need to test them on your own. 

\item
\texttt{convertDouble(double a)}: \texttt{a} will always be a number in the form
.xxx where x is a number. So possible values of \texttt{a} might be $.091, .234,
.123$ so on and so forth. You need to return a type string variable that has
this number. So inputting .123 should return ``.123". You may not use any of the
built in double to string methods for this task.  
 So in short, \texttt{convertDouble} will accept a double  and will
return a string of the same number. 

\item
\texttt{reverseNumber(int x)}: Given \texttt{x}, you need to reverse the number.
For example, if the input is 1992, you need to return 2991. Note that \texttt{x}
will always be less than 10,000 and greater than 1,000. 

\end{enumerate}

\textbf{Note}: All of these problems can be solved without the use of a loop.
You may use a loop if you wish, but know that adding a loop may make the problem
more complex. 

\end{document}

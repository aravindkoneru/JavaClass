\documentclass[11pt,fleqn]{article}

\usepackage[latin1]{inputenc}
\usepackage{enumerate}
\usepackage[hang,flushmargin]{footmisc}
\usepackage{amsmath}
\usepackage{amsfonts}
\usepackage{amssymb}
\usepackage{amsthm}
\usepackage{listings}

\theoremstyle{definition}
\newtheorem{theorem}{Theorem}
\newtheorem{lemma}[theorem]{Lemma}
\newtheorem{corollary}[theorem]{Corollary}
\newtheorem{proposition}[theorem]{Proposition}
\newtheorem{definition}[theorem]{Definition}
\newtheorem{example}[theorem]{Example}

\setlength{\oddsidemargin}{0px}
\setlength{\textwidth}{460px}
\setlength{\voffset}{-1.5cm}
\setlength{\textheight}{20cm}
\setlength{\parindent}{0px}
\setlength{\parskip}{10pt}

\begin{document}
\begin{center}
{\Huge
Homework Assignment 1
}\\
Basic Java and \texttt{if} statements
\end{center}

\textit{Directions}: Please complete as many of the following as possible. If
you can't complete all of them, it's alright. These questions are meant to
be difficult. At the minimum, complete the first 3 questions. Feel free to
collaborate with your peers on the problems, but please note that \underline{all
submitted work must be your own work}. Also if you don't know a certain method,
(for example, you want to know what \texttt{Math.pow(a,b)} is) feel free to look it
up online. That being said, plagiarism will not tolerated. Also as a note to
parents, I would appreciate it if you did not help your children (give them the
answer). Feel free to give them advice, but please do not straight out give them
the answer.    

\begin{enumerate}[Q1.]

\item
\texttt{compareTo(int a, int b)}: Your goal is to compare \texttt{a} to
\texttt{b} and return a number that represents the relationship between the
two. If \texttt{a} is greater than \texttt{b} return 1. If \texttt{a} is less
than \texttt{b}, return -1. If \texttt{a} is equal to \texttt{b} return 0. 

\item
\texttt{evenAndMultiple(int x)}: If a number is both even and a multiple of 7, then
return \texttt{true}. Otherwise, return \texttt{false}.

\item
\texttt{isSpecial(double a)}: A number is special if $\frac{a^2}{3}$ is even or if
$a + \frac{a}{2} * 3$ is even. You may assume that $a<1000$. Test cases will not
be provided for this method and you will need to test them on your own. 

\item
\texttt{calcValue(String a)}: The value of a string is determined by the length
of the string and a variety of other factors. If the length of a string is even,
then the value of the string is calculated by multiplying the length of the
string by 7. If this value is divisible by 3, then divide the value of the
string by 3. If both of these conditions are met, return the value of the
number. Otherwise, just return 0.  

\item
\texttt{reverseNumber(int x)}: Given \texttt{x}, you need to reverse the number.
For example, if the input is 1992, you need to return 2991. Note that \texttt{x}
will always be less than 10,000 and greater than 1,000. 

\end{enumerate}

\textbf{Note}: All of these problems can be solved without the use of a loop.
You may use a loop if you wish, but know that adding a loop may make the problem
more complex. 

\end{document}
